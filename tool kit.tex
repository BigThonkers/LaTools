% packages
\PassOptionsToPackage{dvipsnames}{xcolor}
\documentclass[titlepage,12pt,a4paper,ngerman]{report}
\usepackage[utf8]{inputenc}
\usepackage[T1]{fontenc}
\usepackage[german]{babel}
\usepackage{graphicx}
\usepackage{wrapfig}
\usepackage{amsmath}
\usepackage[hidelinks]{hyperref}
\usepackage{cleveref}
\usepackage{amsfonts}
\usepackage{amssymb}
\usepackage{tikz}
\usetikzlibrary{decorations.pathmorphing,patterns}
\usetikzlibrary{arrows}
\usetikzlibrary{calc}
\usetikzlibrary{decorations.pathreplacing}
\usepackage{tikz-cd}
\usepackage{nicefrac}
\usepackage{mathtools}
\usepackage[margin=1in]{geometry}
\usepackage{enumerate}
\usepackage{tocloft}
\usepackage{tcolorbox}
\setlength{\parskip}{1em}
\usepackage{cancel}
\usepackage{epigraph}
\usepackage{fancyhdr}
\usepackage{bm}
\usepackage[shortlabels]{enumitem}
\usepackage{placeins}
\usepackage{booktabs}
\usepackage{wasysym}

\setlength{\headheight}{35pt}
\setlength\epigraphwidth{.8\textwidth}

% alt math font
\usepackage{eulervm}

% general commands
% zu zeigen symbol
\newcommand{\zz}{\fontfamily{cmss} \selectfont{Z\kern-.61em\raise-0.7ex\hbox{Z}:}}
% build over
\newcommand{\bov}[2]{\buildrel{#2} \over{#1}}
% better looking := (defined as)
\newcommand*{\defeq}{\mathrel{\vcenter{\baselineskip0.5ex \lineskiplimit0pt \hbox{\scriptsize.}\hbox{\scriptsize.}}}=}
\newcommand*{\eqdef}{=\mathrel{\vcenter{\baselineskip0.5ex \lineskiplimit0pt \hbox{\scriptsize.}\hbox{\scriptsize.}}}}
% integral differential d
\newcommand{\dif}{\mathop{}\!\mathrm{d}}
\newcommand{\difi}[1]{\mathrm{d}#1\mathop{}\!}
\newcommand{\dd}{\dif}

\newcommand{\verteq}{\rotatebox{90}{$\,=$}}
\newcommand{\equalto}[2]{\underset{\scriptstyle\overset{\mkern4mu\verteq}{#2}}{#1}}
\newcommand{\equaltoup}[2]{\overset{\scriptstyle\underset{\mkern4mu\verteq}{#2}}{#1}}
\newcommand{\custo}[3]{\underset{\scriptstyle\overset{\mkern4mu\rotatebox{-90}{$\,#1$}}{#3}}{#2}}
\newcommand{\custoup}[3]{\overset{\scriptstyle\overset{\mkern4mu\rotatebox{-90}{$\,#1$}}{#3}}{#2}}
\newcommand{\tx}[1]{\textrm{#1}}
\newcommand{\ov}[1]{\overline{#1}}
\newcommand{\ub}[1]{\underbrace{#1}}
\newcommand{\ob}[1]{\overbrace{#1}}
\newcommand{\const}{\tx{const.}}
\newcommand{\casess}[4]{\left\{ \begin{array}{ll} {#1} & {#2} \\ {#3} & {#4} \end{array} \right.}
\newcommand{\hfw}{\color{RubineRed}\tx{ $\star$hier fehlt was$\star$ } \color{black}}
\newcommand{\prt}[2]{\frac{\partial #1}{\partial #2}}

% linear algebra commands
\def\checkmark{\tikz\fill[scale=0.4](0,.35) -- (.25,0) -- (1,.7) -- (.25,.15) -- cycle;} 
\newcommand{\im}{\tx{im}}
\newcommand{\spa}{\tx{span}}
\newcommand{\adj}{\tx{adj}}
\newcommand{\grad}{\tx{grad}}
\newcommand{\ord}{\tx{ord}}
\newcommand{\basis}[3]{\{#1_{#2}, \dots, #1_{#3}\}}
\newcommand{\dmat}[3]{\begin{pmatrix} #1_{#2}&&\\ &\ddots& \\ && #1_{#3} \end{pmatrix}}
\newcommand{\ska}[2]{\langle #1 , #2 \rangle}

% exp commands
\newcommand{\kq}{\frac{1}{4\pi\epsilon_0}}
\newcommand{\uind}{U_{\tx{ind}}}
\newcommand{\folie}[1]{\color{gray}[Folie: #1]\color{black}}
\newcommand{\versuch}[1]{\color{red!50!black} Versuch: \color{black} #1\\ }

% theo commands
\newcommand{\lag}{\mathcal{L}}
\newcommand{\ham}{\mathcal{H}}
\newcommand{\ham}{\mathcal{H}}
\newcommand{\gre}{\mathcal{G}}
\newcommand{\prt}[2]{\frac{\partial #1}{\partial #2}}
\newcommand{\prd}[2]{\frac{\tx{d} #1}{\tx{d} #2}}
\newcommand{\eofr}{\vec{E}(\vec{r})}
\newcommand{\pofr}{\Phi(\vec{r})}
\renewcommand{\Phi}{\varPhi}
\newcommand{\grr}{\mathcal G(\vec{r},\vec{r}')}
\newcommand{\vphi}{\varphi}

% lab commands
\newcommand\mean{\begin{equation}
\frac{\sum_{i=1}^n x_i}{n}\label{mean}
\end{equation}}
\newcommand\meanstd{\begin{equation}
s_x=\sqrt{\frac{1}{n-1}\sum_{i=1}^n(x_i-\overline{x})^2}\label{meanstd}
\end{equation}}
\newcommand\prodquo{\begin{equation}\left\vert\frac{\Delta z}{z}\right\vert=\sqrt{\left(a\frac{\Delta x}{x}\right)^2+\left(b\frac{\Delta y}{y}\right)^2+\ldots}\textrm{ f\"ur }z=x^a\ y^b\ldots\end{equation}}
\newcommand\tfuncd{\begin{equation}
t=\frac{\vert x_n-y_n\vert}{\sqrt{x_s^2+y_s^2}}
\end{equation}}
\newcommand\tfunc{\begin{equation}
t=\frac{\vert x-y_0\vert}{u_x}
\end{equation}}

% ANDREZ
\newcommand{\summ}[2]{\sum_{#1}^{#2}}
\newcommand{\intt}[2]{\int_{#1}^{#2}}
\newcommand{\lcom}[1]{\color{MidnightBlue}#1\color{black}}
\newcommand{\bei}{\emph{Beispiel: }}
\newcommand{\bem}{\emph{Bemerkung:}}
\renewcommand{\paragraph}[1]{\subsubsection{#1}}
\newcommand{\mau}{$\buildrel \mathcal{O} \over{\textbf{.}}$}

% boxes
\tcbuselibrary{theorems}
\newtcbox{\fribox}[1]{nobeforeafter,colback=white,colframe=red!75!black,fonttitle=\bfseries,title=#1,sharp corners,tcbox raise base} % mahlt eine box nur um den text mit tittel
\newcommand{\frbox}[2]{\begin{tcolorbox}[colback=white,colframe=red!75!black,fonttitle=\bfseries,title=#1]#2\end{tcolorbox}} % mahlt eine große box um alles mit tittel
\newtcbox{\ribox}{nobeforeafter,colback=white,colframe=red!75!black,sharp corners,tcbox raise base} % mahlt eine box nur um den text
\newcommand{\rbox}[1]{\begin{tcolorbox}[colback=white,colframe=red!75!black]#1\end{tcolorbox}} % mahlt eine große box um alles was drinnen ist
\newcommand{\rmbox}[1]{\tcboxmath[colback=white,colframe=red!75!black]{#1}} % mahlt eine box um mathe innerhalb mathmode
\renewcommand{\boxed}{\rmbox}
% super box (looks like regular boxed but wraps around anything)
\newenvironment{supbox}{\begin{tcolorbox}[colback=white,colframe=black,sharp corners,boxrule=.5pt]}{\end{tcolorbox}}
% array type box with title
\newenvironment{zebox}[1]{\begin{array}{|c|}
	\multicolumn{1}{l}{\tx{#1}} \\
	\hline
	\displaystyle
	}{\\ \hline
	\end{array}}
\newcommand{\bbb}[2]{\begin{tcolorbox}[colback=white,colframe=black,fonttitle=\bfseries,title=#1,sharp corners,tcbox raise base]#2\end{tcolorbox}}

% lists
\newlist{arrowlist}{itemize}{1}
\setlist[arrowlist]{label=$\Rightarrow$}

% vector arrow to bold symbol
\renewcommand{\vec}[1]{\bm{#1}}
\newcommand{\vabla}{\vec{\nabla}}

% tikz

\def\centerarc[#1](#2)(#3:#4:#5)% Syntax: [draw options] (center) (initial angle:final angle:radius)
{ \draw[#1] ($(#2)+({#5*cos(#3)},{#5*sin(#3)})$) arc (#3:#4:#5); }

\tikzset{
	annotated cuboid/.pic={
		\tikzset{%
			every edge quotes/.append style={midway, auto},
			/cuboid/.cd,
			#1
		}
		\draw [every edge/.append style={pic actions, densely dashed, opacity=.5}, pic actions]
		(0,0,0) coordinate (o) -- ++(-\cubescale*\cubex,0,0) coordinate (a) -- ++(0,-\cubescale*\cubey,0) coordinate (b) edge coordinate [pos=1] (g) ++(0,0,-\cubescale*\cubez)  -- ++(\cubescale*\cubex,0,0) coordinate (c) -- 
		(o) -- ++(0,0,-\cubescale*\cubez) coordinate (d) -- ++(0,-\cubescale*\cubey,0) coordinate (e) edge (g) -- (c) -- cycle
		(o) -- (a) -- ++(0,0,-\cubescale*\cubez) coordinate (f) edge (g) -- (d) -- cycle;
		\path [every edge/.append style={pic actions, |-|}]
		%(b) +(0,-5pt) coordinate (b1) edge ["\cubex \cubeunits"'] (b1 -| c)
		%(b) +(-5pt,0) coordinate (b2) edge ["\cubey \cubeunits"] (b2 |- a)
		%(c) +(3.5pt,-3.5pt) coordinate (c2) edge ["\cubez \cubeunits"'] ([xshift=3.5pt,yshift=-3.5pt]e)
		;
	},
	/cuboid/.search also={/tikz},
	/cuboid/.cd,
	width/.store in=\cubex,
	height/.store in=\cubey,
	depth/.store in=\cubez,
	units/.store in=\cubeunits,
	scale/.store in=\cubescale,
	width=10,
	height=10,
	depth=10,
	units=cm,
	scale=.1,
}

% other settings
\hbadness=99999
\addtolength{\cftchapnumwidth}{10pt}
\addtolength{\cftsecnumwidth}{10pt}
\addtolength{\cftsubsecnumwidth}{10pt}
\renewcommand{\thechapter}{\Roman{chapter}}
\pagestyle{fancy}

% coloring
%\pagecolor{darkgray}
%\color{white}
\newcommand{\lol}[1]{\color{red}#1\color{black}}
\newcommand{\lal}[1]{\color{yellow}#1\color{black}}
\newcommand{\dod}[1]{\color{green}#1\color{black}}




