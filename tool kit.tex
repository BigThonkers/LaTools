% packages
\PassOptionsToPackage{dvipsnames}{xcolor}
\documentclass[titlepage,12pt,a4paper,ngerman]{report}
\usepackage[utf8]{inputenc}
\usepackage[T1]{fontenc}
\usepackage[german]{babel}
\usepackage{graphicx}
\usepackage{wrapfig}
\usepackage{amsmath}
\usepackage{cleveref}
\usepackage{amsfonts}
\usepackage{amssymb}
\usepackage{tikz}
\usetikzlibrary{decorations.pathmorphing,patterns}
\usetikzlibrary{arrows}
\usepackage{tikz-cd}
\usepackage{nicefrac}
\usepackage{mathtools}
\usepackage[margin=1in]{geometry}
\usepackage{enumerate}
\usepackage{tocloft}
\usepackage{tcolorbox}
\setlength{\parskip}{1em}
\usepackage{cancel}
\usepackage{epigraph}
\usepackage{enumitem}
\usepackage{fancyhdr}

\setlength{\headheight}{35pt}
\setlength\epigraphwidth{.8\textwidth}

% alt math font
\usepackage{eulervm}

% general commands
% zu zeigen symbol
\newcommand{\zz}{\fontfamily{cmss} \selectfont{Z\kern-.61em\raise-0.7ex\hbox{Z}:}}
% build over
\newcommand{\bov}[2]{\buildrel{#2} \over{#1}}
% better looking := (defined as)
\newcommand*{\defeq}{\mathrel{\vcenter{\baselineskip0.5ex \lineskiplimit0pt \hbox{\scriptsize.}\hbox{\scriptsize.}}}=}

\newcommand{\verteq}{\rotatebox{90}{$\,=$}}
\newcommand{\equalto}[2]{\underset{\scriptstyle\overset{\mkern4mu\verteq}{#2}}{#1}}
\newcommand{\equaltoup}[2]{\overset{\scriptstyle\underset{\mkern4mu\verteq}{#2}}{#1}}
\newcommand{\custo}[3]{\underset{\scriptstyle\overset{\mkern4mu\rotatebox{-90}{$\,#1$}}{#3}}{#2}}
\newcommand{\custoup}[3]{\overset{\scriptstyle\overset{\mkern4mu\rotatebox{-90}{$\,#1$}}{#3}}{#2}}
\newcommand{\tx}[1]{\textrm{#1}}
\newcommand{\ov}[1]{\overline{#1}}
\newcommand{\ub}[1]{\underbrace{#1}}
\newcommand{\ob}[1]{\overbrace{#1}}
\newcommand{\const}{\tx{const.}}
\newcommand{\casess}[4]{\left\{ \begin{array}{ll} {#1} & {#2} \\ {#3} & {#4} \end{array} \right.}
\newcommand{\hfw}{\color{RubineRed}\tx{ $\star$hier fehlt was$\star$ } \color{black}}
\newcommand{\prt}[2]{\frac{\partial #1}{\partial #2}}

% linear algebra commands
\def\checkmark{\tikz\fill[scale=0.4](0,.35) -- (.25,0) -- (1,.7) -- (.25,.15) -- cycle;} 
\newcommand{\im}{\tx{im}}
\newcommand{\spa}{\tx{span}}
\newcommand{\adj}{\tx{adj}}
\newcommand{\grad}{\tx{grad}}
\newcommand{\ord}{\tx{ord}}
\newcommand{\basis}[3]{\{#1_{#2}, \dots, #1_{#3}\}}
\newcommand{\dmat}[3]{\begin{pmatrix} #1_{#2}&&\\ &\ddots& \\ && #1_{#3} \end{pmatrix}}
\newcommand{\ska}[2]{\langle #1 , #2 \rangle}

% exp commands
\newcommand{\kq}{\frac{1}{4\pi\epsilon_0}}
\newcommand{\uind}{U_{\tx{ind}}}
\newcommand{\folie}[1]{\color{gray}[Folie: #1]\color{black}}

% theo commands
\newcommand{\lag}{\mathcal{L}}
\newcommand{\ham}{\mathcal{H}}

% boxes
\tcbuselibrary{theorems}
\newtcbox{\fribox}[1]{nobeforeafter,colback=white,colframe=red!75!black,fonttitle=\bfseries,title=#1,sharp corners,tcbox raise base} % mahlt eine box nur um den text mit tittel
\newcommand{\frbox}[2]{\begin{tcolorbox}[colback=white,colframe=red!75!black,fonttitle=\bfseries,title=#1]#2\end{tcolorbox}} % mahlt eine große box um alles mit tittel
\newtcbox{\ribox}{nobeforeafter,colback=white,colframe=red!75!black,sharp corners,tcbox raise base} % mahlt eine box nur um den text
\newcommand{\rbox}[1]{\begin{tcolorbox}[colback=white,colframe=red!75!black]#1\end{tcolorbox}} % mahlt eine große box um alles was drinnen ist
\newcommand{\rmbox}[1]{\tcboxmath[colback=white,colframe=red!75!black]{#1}} % mahlt eine box um mathe innerhalb mathmode
\renewcommand{\boxed}{\rmbox}
% super box (looks like regular boxed but wraps around anything)
\newenvironment{supbox}{\begin{tcolorbox}[colback=white,colframe=black,sharp corners,boxrule=.5pt]}{\end{tcolorbox}}
% array type box with title
\newenvironment{zebox}[1]{\begin{array}{|c|}
	\multicolumn{1}{l}{\tx{#1}} \\
	\hline
	\displaystyle
	}{\\ \hline
	\end{array}}

% vector arrow to bold symbol
\renewcommand{\vec}[1]{\boldsymbol{#1}}

% other settings
\hbadness=99999
\addtolength{\cftchapnumwidth}{10pt}
\addtolength{\cftsecnumwidth}{10pt}
\addtolength{\cftsubsecnumwidth}{10pt}
\renewcommand{\thechapter}{\Roman{chapter}}
\pagestyle{fancy}

% coloring
\pagecolor{darkgray}
\color{white}
\newcommand{\lol}[1]{\color{red}#1\color{black}}
\newcommand{\lal}[1]{\color{yellow}#1\color{black}}
\newcommand{\dod}[1]{\color{green}#1\color{black}}




